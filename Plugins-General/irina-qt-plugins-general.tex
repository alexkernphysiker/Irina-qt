\documentclass[a4paper]{article}
\usepackage [english]{babel}
\usepackage{hyperref}
%
\topmargin -60pt \oddsidemargin -0.4mm \evensidemargin -5.4mm
\setlength{\textheight}{257mm} \setlength{\textwidth}{165mm}
\setlength{\marginparsep}{8mm} \setlength{\marginparwidth}{18mm}
%
\begin{document}
\begin{center}{\textbf{\Large USER'S GUIDE FOR IRINA-QT (base extensions)}}\end{center}
%
\begin{center}\textbf{INTRODUCTION}\end{center}
{
\textbf{Irina-qt is a program for the analysis of data obtained in multi-parametric experiments on nuclear physics.} 
This is the description of the set of basic plugins for this program.
}
%
\begin{center}\textbf{Gnuplot data output extension}\end{center}
{
This module provides export data into text files and plotting them via gnuplot.
It requires \href{http://www.gnuplot.info/index.html}{gnuplot} installed in your system.
This plotting extension can be configured after selecting menu $Results->Data$ $output$ $settings$.
User can specify the path to the gnuplot binary and change some additional parameters.
The most important options are ones that provide saving the plot into image files.
\\
When user exports some spectrum or calculation results set, he must give name to the plotted object.
Two or more objects with the same name cannot be export to text files and plot correctly at the same time.
If there must be two plot lines with the same displayed name, please use $\#$ symbol to create comments which are ignored.
If object name contains $\{line\}$ expression, it is not displayed but gnuplot plots line (otherwise points are plotted).
}
\begin{center}\textbf{Group of datafiles}\end{center}
{
This plugin implements an object that allows to analyse several datafiles as one abstract data source.
It should be added to the project as a single datafile and all datafiles which would be analysed as a group are added into it directly ("Group of DataFiles" tab).
\\
When the analysis runs all these files are read one after another but the events stored there pass through the group object and are analysed like ones produced by a single source.
\\
All variables that are read from datafiles are added to the Group.
If there's name collision the result value can be either a sum or an average value (it's configured).
}
%
\hypertarget{extmasks}{\\}
\begin{center}\textbf{Extended event filter}\end{center}
{
\textbf{Predicate} - is a filter based on conditions '$=$', '$>$', '$<$', '$>=$' or '$<=$'. 
It requires the names of two functions depending on data event and symbol(s) of predicate that describes the condition.
}
\hypertarget{fbkin}{
\begin{center}\textbf{Binary kinematics}\end{center}
}
{
This extension provides object derivative from Function which implements relativistic \textbf{ binary kinematic} formula.
User can use this object for calculating angle of divergence or energy of collision in center of mass system.
Calculating product energy in laboratory system and Jacobian of transformation into center of mass system is also available.
\\ 
All these functions require the same arguments, so are implemented in one type of object.
User has to select the type of returned value.
All arguments are thought to be variables or functions defined in the project.
User cannot write numeric values of any arguments directly in Binary Kinematics form, only variable/function names are acceptable.
If an argument field is empty or contains undeclared variable name it is thought to be equal to zero.
Binary Kinematics object should be named by user for being used as function by other objects of the project.
\\
This module provides also \textbf{Rutherford} formula calculation. 
This is an object derived from Function.
User must input function or variable names for the parameters needed.
All these parameters and their units are signed.
\\
\textbf{Spectrum line} is a binary operator($E_{product}$ MeV, $E_{projectile}$ MeV).
It requires almost the same parameter as binary kinematic formula except projectile energy, which is the second parameter of this operator.
The dependence on the first argument, which is the projectile energy, is the gaussian peak with maximum position calculated using kinematics formula.
Also user should input names of variables or functions which contain the sigma value and total peak amount.
}
%
\hypertarget{errorcalc}{\\}
\begin{center}\textbf{Error calculation}\end{center}
{
{'Error calculation'} is an object derivative from Function which is used for calculation of experimental errors. 
It should be named by user for being used like other functions are. 
It also requires the name of function, which is obtained with error which needs to be calculated, and list of variables which this function depends on and their known errors. 
Zero error value input by user makes Irina-qt to calculate it automatically: firstly it looks for variable {'\textit{$<$name$>$ ERROR}'} ('\textit{$<$name$>$}' is the name of variable, the siffix begins from underline) where the error value is stored, or takes a square root if such variable is not defined.
}
\end{document}